\documentclass[12pt]{article}
% Packages
\usepackage{amsmath, amssymb, amsfonts}
\usepackage{geometry}
\usepackage{enumitem}
\usepackage{fancyhdr}
\usepackage{hyperref}
\usepackage{listings}
\usepackage{xcolor}
\usepackage{graphicx}

%\geometry{a4paper, margin=1in}
%\setlist{nosep}

% Define C++ Code Style
\lstdefinestyle{cppstyle}{
    language=C++,
    basicstyle=\ttfamily\footnotesize,
    keywordstyle=\color{blue},
    commentstyle=\color{green},
    stringstyle=\color{red},
    numbers=left,
    numberstyle=\tiny\color{gray},
    stepnumber=1,
    breaklines=true,
    tabsize=4,
    frame=single
}

\newcommand{\cppinline}[1]{\lstinline[style=cppstyle]|#1|}
\newcommand{\Ito}{It$\hat{\text{o}}$ }
\newcommand{\Itos}{It$\hat{\text{o}}$'s }

% Page setup
\geometry{a4paper, margin=1in}
\pagestyle{fancy}
\fancyhf{}
\lhead{Quant Puzzle Editorial \#002}
\chead{Topic: \Ito Integrals}
\rhead{Difficulty: Interview-Level}
\lfoot{\today}
\cfoot{\thepage}
\rfoot{\textcopyright U. Irons}

\begin{document}

\begin{center}
    \Large \textbf{Quant Puzzle \#002: A Canonical \Ito Integral Puzzle}
\end{center}

\section*{Problem Statement}
\textit{Solve the stochastic integral}
\[
\int_0^t W_s \, dW_s,
\]
\textit{where \(W_t\) is standard Brownian motion.}\\ \\
\section*{Follow-Up Question}
\textit{Find the distribution of}
\[
\int_0^1 W_s \, dW_s.
\]


\bigskip

\section*{Core Ideas}
Key Tools:
\begin{itemize}
    \item Use \Itos lemma on $f(W_t) = \frac{W_t^2}{2}$ to derive a differential involving $W_t \, dW_t$, which rearranges directly into the desired integral.
    \item Recognize that the integral $\int_0^1 W_s \, dW_s$ is a function of \(W_1\), and since \(W_1 \sim \mathcal{N}(0,1)\), the integral's distribution can be derived by a change of variables from a standard normal.
\end{itemize}

\bigskip

\section*{Solution}

We want to compute the stochastic integral
\[
\int_0^t W_s \, dW_s
\]
A naive application of standard calculus might suggest this equals \( \frac{W_t^2}{2} \), since \( \int x \, dx = \frac{x^2}{2} \).  
However, in stochastic calculus, this is incorrect due to the non-zero quadratic variation of Brownian motion.
To get the correct expression, apply \Itos lemma to the function \( f(W_t) = \frac{W_t^2}{2} \). Recall \Itos formula:
\[
df(t,X_t) = \frac{\partial f}{\partial t} \, dt + \frac{\partial f}{\partial x} \, dX_t + \frac{1}{2} \frac{\partial^2 f}{\partial x^2} \, (dX_t)^2
\]
Using \( X_t = W_t \) and \( f(x) = \frac{x^2}{2} \), we get:
\[
d\left(\frac{W_t^2}{2}\right) = W_t \, dW_t + \frac{1}{2} \, dt
\]
Rearranging gives:
\[
W_t \, dW_t = d\left(\frac{W_t^2}{2}\right) - \frac{1}{2} \, dt
\]
Integrate both sides from 0 to \( t \):
\begin{align*}
    \int_0^t W_s \, dW_s &= \frac{W_t^2}{2} - \frac{1}{2} \int_0^t ds = \frac{W_t^2 - t}{2}
\end{align*}
This is the value of the integral.
To find the distribution of \( \int_0^1 W_s \, dW_s \), observe from above:
\[
\int_0^1 W_s \, dW_s = \frac{W_1^2 - 1}{2}
\]
Let \( Z = W_1 \sim \mathcal{N}(0,1) \). Define:
\[
X = \frac{Z^2 - 1}{2}
\]
To find the distribution of \( X \), compute its CDF:
\begin{align*}
    \mathbb{P}(X \leq x) &= \mathbb{P}\left(\frac{Z^2 - 1}{2} \leq x\right) \\
    &= \mathbb{P}\left(Z^2 \leq 2x + 1\right) \\
    &= \mathbb{P}\left(-\sqrt{2x + 1} \leq Z \leq \sqrt{2x + 1}\right) \\
    &= 2 \Phi\left(\sqrt{2x + 1}\right)
\end{align*}
(where \( \Phi \) is the standard normal CDF).
Differentiate to get the density:
\begin{align*}
    f_X(x) &= \frac{d}{dx} \left[ 2 \Phi\left( \sqrt{2x + 1} \right)\right] \\
    &= \frac{2}{\sqrt{2\pi}} \cdot \frac{d}{dx} \left( \int_0^{\sqrt{2x+1}} e^{-y^2/2} \, dy \right) \\
    &= \frac{2}{\sqrt{2\pi}} \cdot \frac{e^{- \frac{2x + 1}{2}}}{2 \sqrt{2x + 1}} \cdot 2 \\
    &= \sqrt{\frac{2}{\pi}} \cdot \frac{e^{- \frac{2x + 1}{2}}}{\sqrt{2x + 1}}
\end{align*}
This is valid only when \( 2x + 1 > 0 \), i.e. \( x > -\frac{1}{2} \). So:
\[
f_X(x) = 
\begin{cases}
    \sqrt{\dfrac{2}{\pi}} \dfrac{e^{-\frac{2x+1}{2}}}{\sqrt{2x+1}}, & x > -\dfrac{1}{2} \\
    0, & \text{otherwise}
\end{cases}
\]
This is the distribution of \( \int_0^1 W_s \, dW_s \).

\bigskip

\section*{Takeaways}
This is a classic quant interview puzzle that tests:

\begin{itemize}
    \item Knowing how and when to apply \Itos lemma, even in reverse.
    \item Comfort with working backwards from stochastic differentials to integrals.
    \item Ability to derive probability distributions via change-of-variable techniques.
    \item Competence in differentiating composite functions involving integrals.
\end{itemize}


\vspace*{\fill}
\begin{center}
    \rule{\linewidth}{0.5pt} \\
    \textbf{More Editorials:} \href{https://github.com/UIronsMaths/Quant-Puzzle-Editorials}{https://github.com/UIronsMaths/Quant-Puzzle-Editorials}
\end{center}

\end{document}