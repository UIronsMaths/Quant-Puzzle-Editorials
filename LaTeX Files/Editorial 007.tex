\documentclass[12pt]{article}
% Packages
\usepackage{amsmath, amssymb, amsfonts}
\usepackage{geometry}
\usepackage{enumitem}
\usepackage{fancyhdr}
\usepackage{hyperref}
\usepackage{listings}
\usepackage{xcolor}
\usepackage{graphicx}

%\geometry{a4paper, margin=1in}
%\setlist{nosep}

% Define C++ Code Style
\lstdefinestyle{cppstyle}{
    language=C++,
    basicstyle=\ttfamily\footnotesize,
    keywordstyle=\color{blue},
    commentstyle=\color{green},
    stringstyle=\color{red},
    numbers=left,
    numberstyle=\tiny\color{gray},
    stepnumber=1,
    breaklines=true,
    tabsize=4,
    frame=single
}

\newcommand{\cppinline}[1]{\lstinline[style=cppstyle]|#1|}
\newcommand{\Ito}{It$\hat{\text{o}}$ }
\newcommand{\Itos}{It$\hat{\text{o}}$'s }
\newcommand{\E}{$\mathbb{E}$}
\newcommand{\Var}{$\mathrm{Var}$}

% Page setup
\geometry{a4paper, margin=1in}
\pagestyle{fancy}
\fancyhf{}
\lhead{Quant Puzzle Editorial \#007}
\chead{Topic: Calculus}
\rhead{Difficulty: Easy}
\lfoot{\today}
\cfoot{\thepage}
\rfoot{\textcopyright U. Irons}

\begin{document}

\begin{center}
    \Large \textbf{Quant Puzzle \#007: Which is Bigger?}
\end{center}

\section*{Problem Statement}
\textit{\textbf{Without} calculating any numerical result, determine which is larger, $e^\pi$ or $\pi^e$.}

\bigskip

\section*{Core Ideas}
We use just two simple tools from calculus and one clever trick involving logarithms to solve this problem:
\begin{itemize}
    \item Logarithms can be used to form an inequality in terms of $\pi$ and $e$.
    \item We can determine if the function on either side of the inequality is increasing and what its maximum is.
    \item We use the quotient rule to determine the first and second derivatives. These are then used to inform on the maxima of the function.
\end{itemize}

\section*{Solution}
\noindent
We begin with the subtle trick that is the most difficult part of solving this problem. By taking logarithms, specifically the natural log, of $e^\pi$ and $\pi^e$ we end up with $\pi\ln(e)$ and $e\ln(\pi)$.

If $e^\pi$ is the larger of the two then we have the inequality $e\ln(\pi)>\pi\ln(e)$. This can be rearranged to $$\frac{\ln(e)}{e}>\frac{\ln(\pi)}{\pi}$$
Alternatively, if $\pi^e$ is the larger of the two, by the same logic we have $$\frac{\ln(e)}{e}<\frac{\ln(\pi)}{\pi}$$
We need to determine which case is true. To do this we notice that each side of both inequalities is a function of the form $$f(x) = \frac{\ln(x)}{x}$$
We also recall from calculus that an increasing function is one where $f(x) < f(y)$ for $x<y$ and a decreasing function is one where $f(x)>f(y)$ for $x<y$. We can use whether or not $f(x)$ is an increasing or decreasing function over the interval $e$ to $\pi$ to determine which inequality holds true.

Using the quotient rule from calculus we see that the derivative of this function is $$f'(x) = \frac{1-\ln(x)}{x^2}$$
Solving $f'(x) = 0$ shows us that we have a stationary point at $x=e$. Considering values of $x$ greater than $e$ yields a negative number in the numerator of $f'(x)$ which implies that $f(x)$ is a decreasing function for $x>e$. For a quick sanity check we consider the second derivative: $$f''(x) = \frac{2\ln(x) - 3}{x^3}$$
which for $x=e$ yields $-\frac{1}{e^3}$ confirming that the stationary point at $x=e$ is indeed a maximum (a global maximum in fact).

To summise, we can now recognise that $$\frac{\ln(e)}{e} > \frac{\ln(\pi)}{\pi}$$
and thus we conclude that $e^\pi$ is the greater of the two.

\section*{Takeaways}
The main difficulty of this problem lies in identifying that taking logarithms can provide a suitable approach. Solving this puzzle proves that you:

\begin{itemize}
    \item Have a strong grasp of basic calculus.
    \item Can re-express a problem into simpler to solve cases (in this case, by using logarithms).
\end{itemize}
This puzzle can be used to test whether or not you can keep calm under pressure when a path to a solution is not obvious. An interviewer will likely hint that a re-expression is required so this puzzle also tests your ability to use the feedback you are being given.

\vspace*{\fill}
\begin{center}
    \rule{\linewidth}{0.5pt} \\
    \textbf{More Editorials:} \href{https://github.com/UIronsMaths/Quant-Puzzle-Editorials}{https://github.com/UIronsMaths/Quant-Puzzle-Editorials}
\end{center}

\end{document}