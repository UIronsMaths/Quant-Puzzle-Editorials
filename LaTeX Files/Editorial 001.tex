\documentclass[12pt]{article}
% Packages
\usepackage{amsmath, amssymb, amsfonts}
\usepackage{geometry}
\usepackage{enumitem}
\usepackage{multicol}
\usepackage{fancyhdr}
\usepackage{textcomp}

% Page setup
\geometry{a4paper, margin=1in}
\pagestyle{fancy}
\fancyhf{}
\lhead{Quant Puzzle Editorial \#001}
\chead{Topic: Brownian Motion}
\rhead{Difficulty: Interview-Level}
\lfoot{\today}
\cfoot{\thepage}
\rfoot{\textcopyright U. Irons}

%\usepackage[a4paper,margin=1in]{geometry} % Adjust margins
\usepackage{hyperref} % Clickable links
%\usepackage{enumitem} % Better lists
\usepackage{listings} % Code formatting
\usepackage{xcolor} % Colors for code
\usepackage{graphicx}

% Define C++ Code Style
\lstdefinestyle{cppstyle}{
    language=C++,
    basicstyle=\ttfamily\footnotesize,
    keywordstyle=\color{blue},
    commentstyle=\color{green},
    stringstyle=\color{red},
    numbers=left,
    numberstyle=\tiny\color{gray},
    stepnumber=1,
    breaklines=true,
    tabsize=4,
    frame=single
}

\newcommand{\cppinline}[1]{\lstinline[style=cppstyle]|#1|}

\begin{document}

\begin{center}
    \Large \textbf{Quant Puzzle \#001: Brownian Flip}
\end{center}

\section*{Problem Statement}
\textit{
Suppose there is a Brownian Motion $W_t$. At time $t=1$, $W_1 > 0$. What is the probability that at time $t=2$, $W_2 < 0$? 
}

\bigskip

\section*{Core Ideas}
The following core ideas are used to solve this problem:
\begin{itemize}
    \item Brownian motion has stationary, independent increments that are normally distributed.
    \item The sign of a Brownian increment is symmetric — positive and negative values are equally likely.
    \item To reach below zero, the downward increment must exceed the positive value at $t=1$ in magnitude.
\end{itemize}

\bigskip

\section*{Solution}
We have two time steps of equal length. The first runs from time $t=0$ to time $t=1$. The second runs from time $t=1$ to $t=2$. We can partition the Brownian motion into these two time intervals to create the Brownian increments $W_2 - W_1$ and $W_1 - W_0$. Brownian increments over arbitrary time intervals $t=x$ to $t=y$, $x\leq y$ are Normally distributed with zero mean and variance equal to $y-x$.
$$W_y - W_x \sim \mathcal{N}(0, y-x), \quad x\leq y$$
Also important is the fact that non-overlapping Brownian increments are independent of each other. Because the increment $W_2 - W_1$ occurs independently of the increment $W_1 - W_0$ it may very well be the case that $W_1 - W_0 > 0$ and $W_2 - W_1 < 0$. The occurrence of one outcome has no effect on the likelihood of the other occurring.\\

In order for $W_2 < 0$ we need $W_2 - W_1 < 0$ which, as we have a symmetric Normal distribution centered around a mean of $0$, occurs with probability $\frac{1}{2}$. We must also have that $|W_2-W_1| > |W_1 - W_0|$. Let $X = W_2 - W_1$ and $Y=W_1 - W_0$. This partitions into two cases:
\begin{enumerate}
    \item $X > Y$ and $Y > 0$
    \item $X < Y$ and $Y < 0$
\end{enumerate}

\begin{align*}
\mathbb{P}(X>Y, Y>0) &= \mathbb{P}(Y>0)\mathbb{P}(X>Y)\\
&= \frac{1}{2}\mathbb{P}(X-Y>0)\\
&= \frac{1}{2}\cdot\frac{1}{2}\\
&= \frac{1}{4}
\end{align*}

And the same result holds for case 2 by identical reasoning. We see then that $\mathbb{P}(|W_2-W_1| > |W_1 - W_0|) = \frac{1}{2}$.
Putting it all together we have:
\begin{align*}
    \mathbb{P}(W_2 < 0) &= \mathbb{P}(\text{sgn}(W_2-W_1) < 0)\cdot\mathbb{P}(|W_2-W_1| > |W_1 - W_0|)\\
    &= \frac{1}{2}\cdot\frac{1}{2}\\
    &= \frac{1}{4}
\end{align*}
which is the solution to this puzzle.

\bigskip

\section*{Takeaways}
This is a classic quant brain-teaser. Simple to state yet subtle to solve. It tests knowledge of:
\begin{itemize}
    \item Brownian increments and their independence
    \item The symmetry of the normal distribution
    \item Conditional probability without conditioning formulas
\end{itemize}
You’ll see variations of this in interviews at hedge funds and trading firms.

\vspace*{\fill}
\begin{center}
    \rule{\linewidth}{0.5pt} \\
    \textbf{More Editorials:} \href{https://github.com/UIronsMaths/Quant-Puzzle-Editorials}{https://github.com/UIronsMaths/Quant-Puzzle-Editorials}
\end{center}

\end{document}
