\documentclass[12pt]{article}
% Packages
\usepackage{amsmath, amssymb, amsfonts}
\usepackage{geometry}
\usepackage{enumitem}
\usepackage{multicol}
\usepackage{fancyhdr}
\usepackage{textcomp}

% Page setup
\geometry{a4paper, margin=1in}
\pagestyle{fancy}
\fancyhf{}
\lhead{Quant Puzzle Editorial \#004}
\chead{Topic: ODEs}
\rhead{Difficulty: Easy}
\lfoot{\today}
\cfoot{\thepage}
\rfoot{\textcopyright U. Irons}

%\usepackage[a4paper,margin=1in]{geometry} % Adjust margins
\usepackage{hyperref} % Clickable links
%\usepackage{enumitem} % Better lists
\usepackage{listings} % Code formatting
\usepackage{xcolor} % Colors for code
\usepackage{graphicx}

% Define C++ Code Style
\lstdefinestyle{cppstyle}{
    language=C++,
    basicstyle=\ttfamily\footnotesize,
    keywordstyle=\color{blue},
    commentstyle=\color{green},
    stringstyle=\color{red},
    numbers=left,
    numberstyle=\tiny\color{gray},
    stepnumber=1,
    breaklines=true,
    tabsize=4,
    frame=single
}

\newcommand{\cppinline}[1]{\lstinline[style=cppstyle]|#1|}

\begin{document}

\begin{center}
    \Large \textbf{Quant Puzzle \#004: An Insightful ODE}
\end{center}

\section*{Problem Statement}
\textit{
Find $f(x)$ if \[f'(x)=f(x)\left(1-f(x)\right)\]
}

\bigskip

\section*{Context \& Key Insights}
This is essentially the logistic equation, a cornerstone in modelling constrained growth. This problem is representative of what you might encounter when modelling population dynamics and could be used in an interview setting as a test of mathematical dexterity and clarity of explanation.

\bigskip

\section*{Core Ideas}
Key Tools:
\begin{itemize}
    \item Use separation of variables ($\frac{dy}{dx}\equiv y'$).
    \item Use Partial Fractions to simplify the integrand of the resulting integral to a form amenable to direct integration.
\end{itemize}

\bigskip

\section*{Solution}
Let $y=f(x)$ for notational simplicity. It is easy to see that:
\[\frac{y'}{y(1-y)}=1\]
Integrating...
\[\int\frac{y'}{y(1-y)}dx = \int 1 dx\]
Note that $\frac{dy}{dx}\equiv y'$ implies that $dy = y' dx$. This is the basis behind the technique of separation of variables. Using $A, B$ and $C$ to denote the arbitrary constants of integration we have
\[\int\frac{1}{y(1-y)}dy = x + A\]
We recognise the denominator of the integrand as a product of two linear factors. Therefore we can use partial fractions to simplify the integrand, making it amenable to direct integration.
\[\int\frac{1}{y(1-y)}dy = \int\frac{1}{y}+\frac{1}{1-y}dy\]
Now that we have something we can integrate we can proceed.
\[\log|y| - \log|1-y| = x + B\]
Using the laws of logarithms we can simplify this to
\[\log\left|\frac{y}{1-y}\right| = x + B\]
Exponentiating and recalling that the exponential function is a purely positive valued function for any input (allowing us to drop the modulus operator) we arrive at
\[\frac{y}{1-y} = Ce^{x}\]
Rearranging for $y$ we get
\[y=\frac{Ce^x}{1+Ce^x}\]
Given the absence of initial or boundary conditions, this is the limit of how much of a solution we can give. We cannot solve for specific values of the arbitrary constant $C$. We can however comment on the asymptotic behaviour of the solution which tends to $1$ as $x\rightarrow{\infty}$.

\bigskip

\section*{Takeaways}
This is a subtle ODE whose difficulty stems from two tricks, one less obvious than the other. These techniques are used to simplify your problem and are:
\begin{itemize}
    \item The use of the separation of variables identity $\frac{dy}{dx}\equiv y'$.
    \item The technique of partial fractions.
\end{itemize}
This problem is classified as Easy, since a mathematically literate postgraduate should be able to solve it by applying standard ODE techniques. The only challenge lies in spotting separation of variables and partial fractions quickly.

\vspace*{\fill}
\begin{center}
    \rule{\linewidth}{0.5pt} \\
    \textbf{More Editorials:} \href{https://github.com/UIronsMaths/Quant-Puzzle-Editorials}{https://github.com/UIronsMaths/Quant-Puzzle-Editorials}
\end{center}

\end{document}